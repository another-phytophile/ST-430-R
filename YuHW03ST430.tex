% Options for packages loaded elsewhere
\PassOptionsToPackage{unicode}{hyperref}
\PassOptionsToPackage{hyphens}{url}
%
\documentclass[
]{article}
\usepackage{amsmath,amssymb}
\usepackage{iftex}
\ifPDFTeX
  \usepackage[T1]{fontenc}
  \usepackage[utf8]{inputenc}
  \usepackage{textcomp} % provide euro and other symbols
\else % if luatex or xetex
  \usepackage{unicode-math} % this also loads fontspec
  \defaultfontfeatures{Scale=MatchLowercase}
  \defaultfontfeatures[\rmfamily]{Ligatures=TeX,Scale=1}
\fi
\usepackage{lmodern}
\ifPDFTeX\else
  % xetex/luatex font selection
\fi
% Use upquote if available, for straight quotes in verbatim environments
\IfFileExists{upquote.sty}{\usepackage{upquote}}{}
\IfFileExists{microtype.sty}{% use microtype if available
  \usepackage[]{microtype}
  \UseMicrotypeSet[protrusion]{basicmath} % disable protrusion for tt fonts
}{}
\makeatletter
\@ifundefined{KOMAClassName}{% if non-KOMA class
  \IfFileExists{parskip.sty}{%
    \usepackage{parskip}
  }{% else
    \setlength{\parindent}{0pt}
    \setlength{\parskip}{6pt plus 2pt minus 1pt}}
}{% if KOMA class
  \KOMAoptions{parskip=half}}
\makeatother
\usepackage{xcolor}
\usepackage[margin=1in]{geometry}
\usepackage{color}
\usepackage{fancyvrb}
\newcommand{\VerbBar}{|}
\newcommand{\VERB}{\Verb[commandchars=\\\{\}]}
\DefineVerbatimEnvironment{Highlighting}{Verbatim}{commandchars=\\\{\}}
% Add ',fontsize=\small' for more characters per line
\usepackage{framed}
\definecolor{shadecolor}{RGB}{248,248,248}
\newenvironment{Shaded}{\begin{snugshade}}{\end{snugshade}}
\newcommand{\AlertTok}[1]{\textcolor[rgb]{0.94,0.16,0.16}{#1}}
\newcommand{\AnnotationTok}[1]{\textcolor[rgb]{0.56,0.35,0.01}{\textbf{\textit{#1}}}}
\newcommand{\AttributeTok}[1]{\textcolor[rgb]{0.13,0.29,0.53}{#1}}
\newcommand{\BaseNTok}[1]{\textcolor[rgb]{0.00,0.00,0.81}{#1}}
\newcommand{\BuiltInTok}[1]{#1}
\newcommand{\CharTok}[1]{\textcolor[rgb]{0.31,0.60,0.02}{#1}}
\newcommand{\CommentTok}[1]{\textcolor[rgb]{0.56,0.35,0.01}{\textit{#1}}}
\newcommand{\CommentVarTok}[1]{\textcolor[rgb]{0.56,0.35,0.01}{\textbf{\textit{#1}}}}
\newcommand{\ConstantTok}[1]{\textcolor[rgb]{0.56,0.35,0.01}{#1}}
\newcommand{\ControlFlowTok}[1]{\textcolor[rgb]{0.13,0.29,0.53}{\textbf{#1}}}
\newcommand{\DataTypeTok}[1]{\textcolor[rgb]{0.13,0.29,0.53}{#1}}
\newcommand{\DecValTok}[1]{\textcolor[rgb]{0.00,0.00,0.81}{#1}}
\newcommand{\DocumentationTok}[1]{\textcolor[rgb]{0.56,0.35,0.01}{\textbf{\textit{#1}}}}
\newcommand{\ErrorTok}[1]{\textcolor[rgb]{0.64,0.00,0.00}{\textbf{#1}}}
\newcommand{\ExtensionTok}[1]{#1}
\newcommand{\FloatTok}[1]{\textcolor[rgb]{0.00,0.00,0.81}{#1}}
\newcommand{\FunctionTok}[1]{\textcolor[rgb]{0.13,0.29,0.53}{\textbf{#1}}}
\newcommand{\ImportTok}[1]{#1}
\newcommand{\InformationTok}[1]{\textcolor[rgb]{0.56,0.35,0.01}{\textbf{\textit{#1}}}}
\newcommand{\KeywordTok}[1]{\textcolor[rgb]{0.13,0.29,0.53}{\textbf{#1}}}
\newcommand{\NormalTok}[1]{#1}
\newcommand{\OperatorTok}[1]{\textcolor[rgb]{0.81,0.36,0.00}{\textbf{#1}}}
\newcommand{\OtherTok}[1]{\textcolor[rgb]{0.56,0.35,0.01}{#1}}
\newcommand{\PreprocessorTok}[1]{\textcolor[rgb]{0.56,0.35,0.01}{\textit{#1}}}
\newcommand{\RegionMarkerTok}[1]{#1}
\newcommand{\SpecialCharTok}[1]{\textcolor[rgb]{0.81,0.36,0.00}{\textbf{#1}}}
\newcommand{\SpecialStringTok}[1]{\textcolor[rgb]{0.31,0.60,0.02}{#1}}
\newcommand{\StringTok}[1]{\textcolor[rgb]{0.31,0.60,0.02}{#1}}
\newcommand{\VariableTok}[1]{\textcolor[rgb]{0.00,0.00,0.00}{#1}}
\newcommand{\VerbatimStringTok}[1]{\textcolor[rgb]{0.31,0.60,0.02}{#1}}
\newcommand{\WarningTok}[1]{\textcolor[rgb]{0.56,0.35,0.01}{\textbf{\textit{#1}}}}
\usepackage{graphicx}
\makeatletter
\def\maxwidth{\ifdim\Gin@nat@width>\linewidth\linewidth\else\Gin@nat@width\fi}
\def\maxheight{\ifdim\Gin@nat@height>\textheight\textheight\else\Gin@nat@height\fi}
\makeatother
% Scale images if necessary, so that they will not overflow the page
% margins by default, and it is still possible to overwrite the defaults
% using explicit options in \includegraphics[width, height, ...]{}
\setkeys{Gin}{width=\maxwidth,height=\maxheight,keepaspectratio}
% Set default figure placement to htbp
\makeatletter
\def\fps@figure{htbp}
\makeatother
\setlength{\emergencystretch}{3em} % prevent overfull lines
\providecommand{\tightlist}{%
  \setlength{\itemsep}{0pt}\setlength{\parskip}{0pt}}
\setcounter{secnumdepth}{-\maxdimen} % remove section numbering
\ifLuaTeX
  \usepackage{selnolig}  % disable illegal ligatures
\fi
\IfFileExists{bookmark.sty}{\usepackage{bookmark}}{\usepackage{hyperref}}
\IfFileExists{xurl.sty}{\usepackage{xurl}}{} % add URL line breaks if available
\urlstyle{same}
\hypersetup{
  pdftitle={Hw03ST430Yu},
  pdfauthor={Haozhe (Jerry) Yu},
  hidelinks,
  pdfcreator={LaTeX via pandoc}}

\title{Hw03ST430Yu}
\author{Haozhe (Jerry) Yu}
\date{2023-10-27}

\begin{document}
\maketitle

\hypertarget{question-1}{%
\section{Question 1}\label{question-1}}

\begin{Shaded}
\begin{Highlighting}[]
\NormalTok{educ }\OtherTok{\textless{}{-}} \FunctionTok{as\_tibble}\NormalTok{(}\FunctionTok{read.table}\NormalTok{(}\StringTok{"https://users.stat.ufl.edu/\textasciitilde{}rrandles/sta4210/Rclassnotes/data/textdatasets/KutnerData/Chapter\%20\%201\%20Data\%20Sets/CH01PR28.txt"}\NormalTok{, }
            \CommentTok{\#sep = "", }
            \AttributeTok{strip.white=}\ConstantTok{TRUE}\NormalTok{,}
            \AttributeTok{col.name =} \FunctionTok{c}\NormalTok{(}\StringTok{"Crime.Rate"}\NormalTok{,}\StringTok{"High.School.Diploma"}\NormalTok{)}
\NormalTok{            ))}
\NormalTok{educ}
\end{Highlighting}
\end{Shaded}

\begin{verbatim}
## # A tibble: 84 x 2
##    Crime.Rate High.School.Diploma
##         <int>               <int>
##  1       8487                  74
##  2       8179                  82
##  3       8362                  81
##  4       8220                  81
##  5       6246                  87
##  6       9100                  66
##  7       6561                  68
##  8       5873                  81
##  9       7993                  74
## 10       7932                  82
## # i 74 more rows
\end{verbatim}

\hypertarget{a.-find-the-least-squares-regression-equation-to-predict-the-crime-rate-from-the-percent-of-individuals-having-at-least-a-high-school-education.-paste-r-or-sas-output-and-then-answer-your-question}{%
\subsection{a. Find the least squares regression equation to predict the
crime rate from the percent of individuals having at least a high school
education. {[}Paste R or SAS output and then answer your
question{]}}\label{a.-find-the-least-squares-regression-equation-to-predict-the-crime-rate-from-the-percent-of-individuals-having-at-least-a-high-school-education.-paste-r-or-sas-output-and-then-answer-your-question}}

\begin{Shaded}
\begin{Highlighting}[]
\NormalTok{educm }\OtherTok{\textless{}{-}} \FunctionTok{lm}\NormalTok{(Crime.Rate}\SpecialCharTok{\textasciitilde{}}\NormalTok{High.School.Diploma,}\AttributeTok{data=}\NormalTok{educ)}
\end{Highlighting}
\end{Shaded}

The equation to predict crime rate (per 100,000 residents) from the
percent of individuals in a country with at least a high school diploma
is

Crime Rate = \ensuremath{2.05176\times 10^{4}} + -170.5751886High School
Percent

\hypertarget{b.-give-the-anova-table-for-this-regression-analysis.-paste-r-or-sas-output}{%
\subsection{b. Give the ANOVA Table for this regression analysis.
{[}Paste R or SAS
output{]}}\label{b.-give-the-anova-table-for-this-regression-analysis.-paste-r-or-sas-output}}

\begin{Shaded}
\begin{Highlighting}[]
\NormalTok{educma }\OtherTok{\textless{}{-}} \FunctionTok{anova}\NormalTok{(educm)}

\NormalTok{educma}
\end{Highlighting}
\end{Shaded}

\begin{verbatim}
## Analysis of Variance Table
## 
## Response: Crime.Rate
##                     Df    Sum Sq  Mean Sq F value    Pr(>F)    
## High.School.Diploma  1  93462942 93462942  16.834 9.571e-05 ***
## Residuals           82 455273165  5552112                      
## ---
## Signif. codes:  0 '***' 0.001 '**' 0.01 '*' 0.05 '.' 0.1 ' ' 1
\end{verbatim}

\hypertarget{c.-find-sse-and-mse-for-this-model.}{%
\subsection{c.~Find SSE and MSE for this
model.}\label{c.-find-sse-and-mse-for-this-model.}}

The SSE for this model is \ensuremath{4.5527317\times 10^{8}} and the
MSE is \ensuremath{5.5521118\times 10^{6}}.

\hypertarget{d.-what-is-the-estimate-of-sigma-from-this-analysis}{%
\subsection{d.~What is the estimate of sigma from this
analysis?}\label{d.-what-is-the-estimate-of-sigma-from-this-analysis}}

The estimate of \(\sigma\) for this analysis is 2356.2919539

\hypertarget{e.-what-percent-of-the-variation-in-crime-rates-can-be-explained-by-the-percent-of-high-school-graduates}{%
\subsection{e. What percent of the variation in crime rates can be
explained by the percent of high school
graduates?}\label{e.-what-percent-of-the-variation-in-crime-rates-can-be-explained-by-the-percent-of-high-school-graduates}}

The percent of variation in crime rates explained by the percent of high
school grads is 0.170324

\hypertarget{f.-what-is-the-correlation-between-crime-rates-and-percent-of-high-school-graduates}{%
\subsection{f.~What is the correlation between crime rates and percent
of high school
graduates?}\label{f.-what-is-the-correlation-between-crime-rates-and-percent-of-high-school-graduates}}

The correlation between crime rates and percent of high school graduates
is -0.4127033

\hypertarget{g.-based-on-your-anova-table-is-the-linear-relationship-between-x-and-y-statistically-significant-be-sure-to-give-an-appropriate-null-and-alternate-hypothesis-test-statistic-its-associated-degrees-of-freedom-and-the-p-value.}{%
\subsection{g. Based on your ANOVA table, is the linear relationship
between X and Y statistically significant? Be sure to give an
appropriate null and alternate hypothesis, test statistic, its
associated degrees of freedom, and the
p-value.}\label{g.-based-on-your-anova-table-is-the-linear-relationship-between-x-and-y-statistically-significant-be-sure-to-give-an-appropriate-null-and-alternate-hypothesis-test-statistic-its-associated-degrees-of-freedom-and-the-p-value.}}

\begin{itemize}
\item
  H0: There is no linear relationship between crime rates and percent of
  high school graduates (\(\beta_1\) =0)
\item
  HA: There is a linear relationship between crime rates and percent of
  high school graduates (\(\beta_1\) ne 0)
\item
  Test Statistic (F value): 16.8337645
\item
  Degrees of Freedom: 1 for the model (High School Diploma Percent), and
  82 for the error.
\item
  P value: \ensuremath{9.5713958\times 10^{-5}}
\end{itemize}

As p \textless{} 0.05, we reject H0 at \(\alpha\) = 0.05 and conclude
that there is statistically significant evidence for a linear
relationship between crime rate and percent of high school graduates.

\hypertarget{h.-give-a-scatter-plot-of-crime-rates-vs.-percent-of-high-school-graduates-with-the-regression-line.-comment-about-linearity}{%
\subsection{h. Give a scatter plot of crime rates vs.~percent of high
school graduates, with the regression line. Comment about
linearity}\label{h.-give-a-scatter-plot-of-crime-rates-vs.-percent-of-high-school-graduates-with-the-regression-line.-comment-about-linearity}}

\begin{Shaded}
\begin{Highlighting}[]
\FunctionTok{ggplot}\NormalTok{(educ,}\FunctionTok{aes}\NormalTok{(}\AttributeTok{x=}\NormalTok{High.School.Diploma,}\AttributeTok{y=}\NormalTok{Crime.Rate))}\SpecialCharTok{+}
  \FunctionTok{geom\_jitter}\NormalTok{(}\AttributeTok{color=}\StringTok{"turquoise"}\NormalTok{)}\SpecialCharTok{+}
  \FunctionTok{geom\_smooth}\NormalTok{(}\AttributeTok{method=}\StringTok{\textquotesingle{}lm\textquotesingle{}}\NormalTok{, }\AttributeTok{formula=}\NormalTok{ y}\SpecialCharTok{\textasciitilde{}}\NormalTok{x, }
              \AttributeTok{se=}\ConstantTok{FALSE}\NormalTok{,}
              \AttributeTok{show.legend=}\ConstantTok{TRUE}\NormalTok{)}\SpecialCharTok{+}
    \FunctionTok{stat\_poly\_eq}\NormalTok{(}\AttributeTok{eq.with.lhs =} \StringTok{"italic(hat(y))\textasciitilde{}\textasciigrave{}=\textasciigrave{}\textasciitilde{}"}\NormalTok{,}
               \FunctionTok{use\_label}\NormalTok{(}\FunctionTok{c}\NormalTok{(}\StringTok{"eq"}\NormalTok{, }\StringTok{"R2"}\NormalTok{)))}\SpecialCharTok{+}
  \FunctionTok{labs}\NormalTok{(}\AttributeTok{title =} \FunctionTok{paste}\NormalTok{(}\StringTok{"Q1: Scatterplot of Crime Rate and Percent of High School Graduates }\SpecialCharTok{\textbackslash{}n}\StringTok{ with Linear Regression Line and Equation"}\NormalTok{),}
         \AttributeTok{subtitle =} \StringTok{"by Jerry Yu"}\NormalTok{)}\SpecialCharTok{+} 
  \FunctionTok{theme}\NormalTok{(}\AttributeTok{plot.title =} \FunctionTok{element\_text}\NormalTok{(}\AttributeTok{size =} \DecValTok{12}\NormalTok{))}
\end{Highlighting}
\end{Shaded}

\includegraphics{YuHW03ST430_files/figure-latex/scatter1-1.pdf} As there
does not seem to be a nonlinear pattern in the scatterplot, and the
regression line seems to slice across the residuals equally, leaving
about 1/2 above and below. I would say that the data seems linear.

\hypertarget{i.-give-the-residual-plot-residuals-vs.-fitted-values.-test-for-non-linear-and-non-constant-variance.}{%
\subsection{i. Give the Residual Plot (residuals vs.~fitted values).
Test for Non-Linear and Non-constant
variance.}\label{i.-give-the-residual-plot-residuals-vs.-fitted-values.-test-for-non-linear-and-non-constant-variance.}}

\begin{Shaded}
\begin{Highlighting}[]
\NormalTok{educmr }\OtherTok{\textless{}{-}} \FunctionTok{tibble}\NormalTok{(}
  \StringTok{"fit"} \OtherTok{=}\NormalTok{ educm}\SpecialCharTok{$}\NormalTok{fitted.values,}
  \StringTok{"resid"} \OtherTok{=}\NormalTok{ educm}\SpecialCharTok{$}\NormalTok{residuals}
\NormalTok{)}

\FunctionTok{ggplot}\NormalTok{(educmr,}\FunctionTok{aes}\NormalTok{(}\AttributeTok{x=}\NormalTok{fit,}\AttributeTok{y=}\NormalTok{resid))}\SpecialCharTok{+}
  \FunctionTok{geom\_jitter}\NormalTok{(}\AttributeTok{color=}\StringTok{"salmon"}\NormalTok{)}\SpecialCharTok{+}
  \FunctionTok{geom\_hline}\NormalTok{(}\AttributeTok{yintercept =} \DecValTok{0}\NormalTok{, }\AttributeTok{linetype=}\StringTok{"dotted"}\NormalTok{)}\SpecialCharTok{+}
  \FunctionTok{labs}\NormalTok{(}\AttributeTok{title =} \FunctionTok{paste}\NormalTok{(}\StringTok{"Q1: Residuals Versus Fitted Values for the Educ Data Set"}\NormalTok{),}
         \AttributeTok{subtitle =} \StringTok{"by Jerry Yu"}\NormalTok{)}\SpecialCharTok{+} 
  \FunctionTok{theme}\NormalTok{(}\AttributeTok{plot.title =} \FunctionTok{element\_text}\NormalTok{(}\AttributeTok{size =} \DecValTok{14}\NormalTok{))}
\end{Highlighting}
\end{Shaded}

\includegraphics{YuHW03ST430_files/figure-latex/fitplot1-1.pdf}

As there do not seem to be patterns in the distribution of the
residuals, nor any fan and funnel shapes, I conclude that the variance
is likely linear and constant.

\hypertarget{j.-conduct-breusch-pagan-test-for-the-constancy-of-the-error-variance.-be-sure-to-give-an-appropriate-null-and-alternate-hypothesis-test-statistic-its-associated-degrees-of-freedom-and-the-p-value.}{%
\subsection{j. Conduct Breusch-Pagan Test for the constancy of the error
variance. Be sure to give an appropriate null and alternate hypothesis,
test statistic, its associated degrees of freedom, and the
p-value.}\label{j.-conduct-breusch-pagan-test-for-the-constancy-of-the-error-variance.-be-sure-to-give-an-appropriate-null-and-alternate-hypothesis-test-statistic-its-associated-degrees-of-freedom-and-the-p-value.}}

\begin{Shaded}
\begin{Highlighting}[]
\NormalTok{educmbp }\OtherTok{\textless{}{-}} \FunctionTok{bptest}\NormalTok{(educm,}\AttributeTok{studentize =} \ConstantTok{FALSE}\NormalTok{)}
\NormalTok{educmbp}
\end{Highlighting}
\end{Shaded}

\begin{verbatim}
## 
##  Breusch-Pagan test
## 
## data:  educm
## BP = 0.005045, df = 1, p-value = 0.9434
\end{verbatim}

\begin{Shaded}
\begin{Highlighting}[]
\FunctionTok{ncvTest}\NormalTok{(educm)}
\end{Highlighting}
\end{Shaded}

\begin{verbatim}
## Non-constant Variance Score Test 
## Variance formula: ~ fitted.values 
## Chisquare = 0.005045022, Df = 1, p = 0.94338
\end{verbatim}

\begin{itemize}
\item
  H0: Equal Variance Among Errors
\item
  HA: Unequal Variance Among Errors
\item
  Degree of Freedom: 1
\item
  P Value: 0.9433752
\end{itemize}

\hypertarget{k.-index-plot-to-test-for-independence-of-errors.}{%
\subsection{k. Index Plot to test for Independence of
errors.}\label{k.-index-plot-to-test-for-independence-of-errors.}}

\begin{Shaded}
\begin{Highlighting}[]
\FunctionTok{ggplot}\NormalTok{(educmr, }\FunctionTok{aes}\NormalTok{(}\AttributeTok{x =} \DecValTok{1}\SpecialCharTok{:}\FunctionTok{length}\NormalTok{(resid), }\AttributeTok{y =}\NormalTok{ resid)) }\SpecialCharTok{+}
  \FunctionTok{geom\_point}\NormalTok{(}\AttributeTok{color =} \StringTok{"aquamarine"}\NormalTok{) }\SpecialCharTok{+}
  \FunctionTok{labs}\NormalTok{(}\AttributeTok{x =} \StringTok{"Index"}\NormalTok{,}
       \AttributeTok{title =} \StringTok{"Q1 Residual Time Sequence Plot for Educ Data"}\NormalTok{,}
       \AttributeTok{subtitle =} \StringTok{"by Jerry Yu"}\NormalTok{) }\SpecialCharTok{+}
  \FunctionTok{geom\_hline}\NormalTok{(}\AttributeTok{yintercept =} \DecValTok{0}\NormalTok{,}
             \AttributeTok{color =} \StringTok{"darkblue"}\NormalTok{,}
             \AttributeTok{linetype =} \StringTok{"dotdash"}\NormalTok{)}
\end{Highlighting}
\end{Shaded}

\includegraphics{YuHW03ST430_files/figure-latex/unnamed-chunk-1-1.pdf}

\hypertarget{l.-conduct-durbin-watson-test.-be-sure-to-give-an-appropriate-null-and-alternate-hypothesis-test-statistic-and-the-p-value.}{%
\subsection{l. Conduct Durbin-Watson Test. Be sure to give an
appropriate null and alternate hypothesis, test statistic and the
p-value.}\label{l.-conduct-durbin-watson-test.-be-sure-to-give-an-appropriate-null-and-alternate-hypothesis-test-statistic-and-the-p-value.}}

\begin{Shaded}
\begin{Highlighting}[]
\FunctionTok{dwtest}\NormalTok{(educm)}
\end{Highlighting}
\end{Shaded}

\begin{verbatim}
## 
##  Durbin-Watson test
## 
## data:  educm
## DW = 1.4951, p-value = 0.008696
## alternative hypothesis: true autocorrelation is greater than 0
\end{verbatim}

\begin{Shaded}
\begin{Highlighting}[]
\NormalTok{educmw }\OtherTok{\textless{}{-}}\FunctionTok{durbinWatsonTest}\NormalTok{(educm)}
\NormalTok{educmw}
\end{Highlighting}
\end{Shaded}

\begin{verbatim}
##  lag Autocorrelation D-W Statistic p-value
##    1         0.25204      1.495148   0.022
##  Alternative hypothesis: rho != 0
\end{verbatim}

\begin{itemize}
\item
  H0: Errors are uncorrelated over time
\item
  HA: Errors are correlated (either positive or negative). I used the
  \texttt{car} test where the alternative hypothesis is 2 sided
\item
  Test Statistic: 1.4951485
\item
  p value: 0.022
\end{itemize}

\hypertarget{m.-outlier-deduction-test-plot-standardized-residuals-versus-fitted-values}{%
\subsection{m. Outlier deduction test {[}Plot standardized Residuals
versus fitted
values{]}}\label{m.-outlier-deduction-test-plot-standardized-residuals-versus-fitted-values}}

\begin{Shaded}
\begin{Highlighting}[]
\NormalTok{educmrs }\OtherTok{\textless{}{-}} \FunctionTok{add\_column}\NormalTok{(educmr, }\StringTok{"rstandardized"}\OtherTok{=}\FunctionTok{rstandard}\NormalTok{(educm))}

\FunctionTok{ggplot}\NormalTok{(educmrs, }\FunctionTok{aes}\NormalTok{(}\AttributeTok{x =}\NormalTok{ fit, }\AttributeTok{y =}\NormalTok{ rstandardized)) }\SpecialCharTok{+}
  \FunctionTok{geom\_point}\NormalTok{(}\AttributeTok{color =} \StringTok{"lightgreen"}\NormalTok{) }\SpecialCharTok{+}
  \FunctionTok{labs}\NormalTok{(}\AttributeTok{x =} \StringTok{"Index"}\NormalTok{,}
       \AttributeTok{title =} \StringTok{"Q1 Outlier Detection Plot with Standarized Residuals vs Fit"}\NormalTok{,}
       \AttributeTok{subtitle =} \StringTok{"by Jerry Yu"}\NormalTok{) }\SpecialCharTok{+}
  \FunctionTok{geom\_hline}\NormalTok{(}\AttributeTok{yintercept =} \DecValTok{0}\NormalTok{,}
             \AttributeTok{color =} \StringTok{"darkblue"}\NormalTok{,}
             \AttributeTok{linetype =} \StringTok{"solid"}\NormalTok{) }\SpecialCharTok{+}
    \FunctionTok{geom\_hline}\NormalTok{(}\AttributeTok{yintercept =} \SpecialCharTok{{-}}\DecValTok{2}\NormalTok{,}
             \AttributeTok{color =} \StringTok{"blue"}\NormalTok{,}
             \AttributeTok{linetype =} \StringTok{"dashed"}\NormalTok{)}\SpecialCharTok{+}
  \FunctionTok{geom\_hline}\NormalTok{(}\AttributeTok{yintercept =} \DecValTok{2}\NormalTok{,}
             \AttributeTok{color =} \StringTok{"blue"}\NormalTok{,}
             \AttributeTok{linetype =} \StringTok{"dashed"}\NormalTok{)}
\end{Highlighting}
\end{Shaded}

\includegraphics{YuHW03ST430_files/figure-latex/outlier1-1.pdf}

\begin{Shaded}
\begin{Highlighting}[]
\NormalTok{educmro }\OtherTok{\textless{}{-}} \FunctionTok{filter}\NormalTok{(educmrs,}\FunctionTok{abs}\NormalTok{(rstandardized) }\SpecialCharTok{\textgreater{}}\DecValTok{2}\NormalTok{)}
\NormalTok{educmro}
\end{Highlighting}
\end{Shaded}

\begin{verbatim}
## # A tibble: 2 x 3
##     fit  resid rstandardized
##   <dbl>  <dbl>         <dbl>
## 1 7213.  6803.          2.90
## 2 7383. -5278.         -2.25
\end{verbatim}

We have 2 outliers, one where the fit = 7212.7352313 and 7383.31042

\hypertarget{n.-give-a-histogram-of-the-residuals-and-the-density-curve.-comment-about-the-distribution-of-residuals.}{%
\subsection{n.~Give a Histogram of the residuals and the density curve.
Comment about the distribution of
residuals.}\label{n.-give-a-histogram-of-the-residuals-and-the-density-curve.-comment-about-the-distribution-of-residuals.}}

\begin{Shaded}
\begin{Highlighting}[]
\FunctionTok{ggplot}\NormalTok{(}\AttributeTok{data =}\NormalTok{ educmrs, }\FunctionTok{aes}\NormalTok{(}\AttributeTok{x =}\NormalTok{ resid, }\AttributeTok{y =} \FunctionTok{after\_stat}\NormalTok{(density))) }\SpecialCharTok{+}
  \FunctionTok{geom\_histogram}\NormalTok{(}\AttributeTok{fill =} \StringTok{"lightblue"}\NormalTok{) }\SpecialCharTok{+}
  \FunctionTok{geom\_density}\NormalTok{(}\AttributeTok{color =} \StringTok{"darkgreen"}\NormalTok{) }\SpecialCharTok{+}
  \FunctionTok{labs}\NormalTok{(}
    \AttributeTok{title =} \FunctionTok{paste}\NormalTok{(}\StringTok{"Histogram and Density Plot of Residuals for Data Set Educ"}\NormalTok{),}
    \AttributeTok{subtitle =} \StringTok{"by Jerry Yu"}
\NormalTok{  ) }\SpecialCharTok{+}
  \FunctionTok{ylab}\NormalTok{(}\StringTok{"Density of Residuals"}\NormalTok{)}
\end{Highlighting}
\end{Shaded}

\begin{verbatim}
## `stat_bin()` using `bins = 30`. Pick better value with `binwidth`.
\end{verbatim}

\includegraphics{YuHW03ST430_files/figure-latex/density1-1.pdf}

There seems to be a slight right skew in the data, The 2 outliers
detected in part m are clearly visible.

\hypertarget{o.-give-a-qq-plot-of-the-residuals-to-test-for-normality-of-error-terms.-comment-about-the-distribution-of-residuals.}{%
\subsection{o. Give a QQ-plot of the residuals to test for normality of
error terms. Comment about the distribution of
residuals.}\label{o.-give-a-qq-plot-of-the-residuals-to-test-for-normality-of-error-terms.-comment-about-the-distribution-of-residuals.}}

\begin{Shaded}
\begin{Highlighting}[]
\FunctionTok{ggplot}\NormalTok{(}\AttributeTok{data =}\NormalTok{ educmrs, }\FunctionTok{aes}\NormalTok{(}\AttributeTok{sample =}\NormalTok{ resid))}\SpecialCharTok{+}
  \FunctionTok{geom\_qq}\NormalTok{( }\AttributeTok{color=}\StringTok{"coral"}\NormalTok{)}\SpecialCharTok{+}
  \FunctionTok{geom\_qq\_line}\NormalTok{( }\AttributeTok{color=}\StringTok{"turquoise"}\NormalTok{)}\SpecialCharTok{+}
  \FunctionTok{labs}\NormalTok{(}
    \AttributeTok{title =} \FunctionTok{paste}\NormalTok{(}\StringTok{"Q1: QQ Plot of Residuals for Educ Linear Regression Model"}\NormalTok{),}
    \AttributeTok{subtitle =} \StringTok{"by Jerry Yu"}
\NormalTok{  ) }\SpecialCharTok{+}
  \FunctionTok{xlab}\NormalTok{(}\StringTok{"Theoretical Quantiles"}\NormalTok{)}\SpecialCharTok{+}
  \FunctionTok{ylab}\NormalTok{(}\StringTok{"Sample Quantiles"}\NormalTok{)}
\end{Highlighting}
\end{Shaded}

\includegraphics{YuHW03ST430_files/figure-latex/qq1-1.pdf}

The data visually does not look normal, as the extreme Residuals both
look flatter than the Theoretical Residuals (the line).

\hypertarget{p.-conduct-a-shapiro-wilk-test-on-the-residuals.-be-sure-to-give-an-appropriate-null-and-alternate-hypothesis-test-statistic-and-the-p-value.-give-the-p-value-for-this-test-and-explain-what-this-means-in-terms-of-our-model-assumptions.}{%
\subsection{p.~Conduct a Shapiro-Wilk Test on the residuals. Be sure to
give an appropriate null and alternate hypothesis, test statistic and
the p-value. Give the p-value for this test and explain what this means
in terms of our model
assumptions.}\label{p.-conduct-a-shapiro-wilk-test-on-the-residuals.-be-sure-to-give-an-appropriate-null-and-alternate-hypothesis-test-statistic-and-the-p-value.-give-the-p-value-for-this-test-and-explain-what-this-means-in-terms-of-our-model-assumptions.}}

\begin{Shaded}
\begin{Highlighting}[]
\NormalTok{shap1 }\OtherTok{\textless{}{-}} \FunctionTok{shapiro.test}\NormalTok{(educmrs}\SpecialCharTok{$}\NormalTok{resid)}
\NormalTok{shap1}
\end{Highlighting}
\end{Shaded}

\begin{verbatim}
## 
##  Shapiro-Wilk normality test
## 
## data:  educmrs$resid
## W = 0.97763, p-value = 0.1515
\end{verbatim}

\begin{itemize}
\item
  H0: The random error is normally distributed
\item
  Ha: The random error is not normally distributed
\item
  Test Statistic: 0.9776328
\item
  p value: 0.1514916
\end{itemize}

As p \textgreater{} 0.05, we fail to reject H0 at \(\alpha\) = 0.05 and
conclude that there is no statistically significant evidence that the
random error is not normally distributed.

\hypertarget{question-2}{%
\section{Question 2}\label{question-2}}

\begin{quote}
2.Download the ``Explosives dataset'' from Moodle. Fit a simple linear
regression, relating the deflection of galvonometer (Y) to the area of
the wires on the coupling (X). Complete the following parts.
\end{quote}

\begin{Shaded}
\begin{Highlighting}[]
\NormalTok{xplode }\OtherTok{\textless{}{-}} \FunctionTok{as\_tibble}\NormalTok{(}\FunctionTok{read.table}\NormalTok{(}\StringTok{"Datasets/explosives.txt"}\NormalTok{, }
            \AttributeTok{strip.white=}\ConstantTok{TRUE}\NormalTok{,}
            \AttributeTok{col.name =} \FunctionTok{c}\NormalTok{(}\StringTok{"Coupling.Number"}\NormalTok{,}\StringTok{"Wire.Area"}\NormalTok{,}\StringTok{"Galvonometer"}\NormalTok{)}
\NormalTok{            ))}
\NormalTok{xplode}
\end{Highlighting}
\end{Shaded}

\begin{verbatim}
## # A tibble: 22 x 3
##    Coupling.Number Wire.Area Galvonometer
##              <int>     <int>        <dbl>
##  1               1       152         85  
##  2               1       152         81.5
##  3               1       152         83.5
##  4               2       125        102  
##  5               2       125         90.5
##  6               2       125         98.5
##  7               3        99        109  
##  8               3        99        116. 
##  9               4        66        131  
## 10               4        66        128. 
## # i 12 more rows
\end{verbatim}

\hypertarget{a.-give-a-scatter-plot}{%
\subsection{a. Give a scatter plot}\label{a.-give-a-scatter-plot}}

\begin{Shaded}
\begin{Highlighting}[]
\FunctionTok{ggplot}\NormalTok{(xplode,}\FunctionTok{aes}\NormalTok{(}\AttributeTok{x=}\NormalTok{Wire.Area,}\AttributeTok{y=}\NormalTok{Galvonometer))}\SpecialCharTok{+}
  \FunctionTok{geom\_jitter}\NormalTok{(}\AttributeTok{color=}\StringTok{"salmon"}\NormalTok{)}\SpecialCharTok{+}
  \FunctionTok{labs}\NormalTok{(}\AttributeTok{title =} \FunctionTok{paste}\NormalTok{(}\StringTok{"Scatterplot of Wire Area and Galvonometer Deflection"}\NormalTok{),}
         \AttributeTok{subtitle =} \StringTok{"by Jerry Yu"}\NormalTok{)}
\end{Highlighting}
\end{Shaded}

\includegraphics{YuHW03ST430_files/figure-latex/scatter2-1.pdf}

\hypertarget{b.-find-the-least-squares-regression.}{%
\subsection{b. Find the least squares
regression.}\label{b.-find-the-least-squares-regression.}}

\begin{Shaded}
\begin{Highlighting}[]
\NormalTok{xplodem }\OtherTok{\textless{}{-}} \FunctionTok{lm}\NormalTok{(Galvonometer}\SpecialCharTok{\textasciitilde{}}\NormalTok{Wire.Area,xplode)}
\NormalTok{xplodem}
\end{Highlighting}
\end{Shaded}

\begin{verbatim}
## 
## Call:
## lm(formula = Galvonometer ~ Wire.Area, data = xplode)
## 
## Coefficients:
## (Intercept)    Wire.Area  
##    184.4357      -0.6954
\end{verbatim}

\hypertarget{c.-give-the-residual-plot-residuals-vs.-fitted-values.-test-for-non-linear-and-non-constant-variance.}{%
\subsection{c.~Give the Residual Plot (residuals vs.~fitted values).
Test for Non-Linear and Non-constant
variance.}\label{c.-give-the-residual-plot-residuals-vs.-fitted-values.-test-for-non-linear-and-non-constant-variance.}}

\begin{Shaded}
\begin{Highlighting}[]
\NormalTok{xplodemr }\OtherTok{\textless{}{-}} \FunctionTok{tibble}\NormalTok{(}
  \StringTok{"fit"} \OtherTok{=}\NormalTok{ xplodem}\SpecialCharTok{$}\NormalTok{fitted.values,}
  \StringTok{"resid"} \OtherTok{=}\NormalTok{ xplodem}\SpecialCharTok{$}\NormalTok{residuals}
\NormalTok{)}

\FunctionTok{ggplot}\NormalTok{(xplodemr,}\FunctionTok{aes}\NormalTok{(}\AttributeTok{x=}\NormalTok{fit,}\AttributeTok{y=}\NormalTok{resid))}\SpecialCharTok{+}
  \FunctionTok{geom\_jitter}\NormalTok{(}\AttributeTok{color=}\StringTok{"yellowgreen"}\NormalTok{)}\SpecialCharTok{+}
  \FunctionTok{geom\_hline}\NormalTok{(}\AttributeTok{yintercept =} \DecValTok{0}\NormalTok{, }\AttributeTok{linetype=}\StringTok{"dotted"}\NormalTok{)}\SpecialCharTok{+}
  \FunctionTok{labs}\NormalTok{(}\AttributeTok{title =} \FunctionTok{paste}\NormalTok{(}\StringTok{"Residuals Versus Fitted Values for the Xplode Data Set"}\NormalTok{),}
         \AttributeTok{subtitle =} \StringTok{"by Jerry Yu"}\NormalTok{)}\SpecialCharTok{+} 
  \FunctionTok{theme}\NormalTok{(}\AttributeTok{plot.title =} \FunctionTok{element\_text}\NormalTok{(}\AttributeTok{size =} \DecValTok{12}\NormalTok{))}
\end{Highlighting}
\end{Shaded}

\includegraphics{YuHW03ST430_files/figure-latex/fitplot2-1.pdf}

There does seem to be a pattern the the distribution of residuals and
fitted values, with mostly positive residuals between 75 and 100 and
160-175, and mostly negative between 100 and 160. This might indicate
that the variance is not linear. However, the pattern of the variances
does not assume the shape of a funnel, so there is no evidence that the
variance is non constant. However, as there are relatively few data
points, we cannot conclude anything from the residual plot, and will
rely on the Breusch-Pagan Test in part d.~

\hypertarget{d.-conduct-breusch-pagan-test-for-the-constancy-of-the-error-variance.}{%
\subsection{d.~Conduct Breusch-Pagan Test for the constancy of the error
variance.}\label{d.-conduct-breusch-pagan-test-for-the-constancy-of-the-error-variance.}}

\begin{Shaded}
\begin{Highlighting}[]
\NormalTok{xplodembp }\OtherTok{\textless{}{-}} \FunctionTok{bptest}\NormalTok{(xplodem,}\AttributeTok{studentize =} \ConstantTok{FALSE}\NormalTok{)}
\NormalTok{xplodembp}
\end{Highlighting}
\end{Shaded}

\begin{verbatim}
## 
##  Breusch-Pagan test
## 
## data:  xplodem
## BP = 3.6375, df = 1, p-value = 0.05649
\end{verbatim}

\begin{Shaded}
\begin{Highlighting}[]
\FunctionTok{ncvTest}\NormalTok{(xplodem)}
\end{Highlighting}
\end{Shaded}

\begin{verbatim}
## Non-constant Variance Score Test 
## Variance formula: ~ fitted.values 
## Chisquare = 3.637537, Df = 1, p = 0.05649
\end{verbatim}

\hypertarget{e.-index-plot-to-test-for-independence-of-errors.}{%
\subsection{e. Index Plot to test for Independence of
errors.}\label{e.-index-plot-to-test-for-independence-of-errors.}}

\begin{Shaded}
\begin{Highlighting}[]
\FunctionTok{ggplot}\NormalTok{(xplodemr, }\FunctionTok{aes}\NormalTok{(}\AttributeTok{x =} \DecValTok{1}\SpecialCharTok{:}\FunctionTok{length}\NormalTok{(resid), }\AttributeTok{y =}\NormalTok{ resid)) }\SpecialCharTok{+}
  \FunctionTok{geom\_point}\NormalTok{(}\AttributeTok{color =} \StringTok{"aquamarine"}\NormalTok{) }\SpecialCharTok{+}
  \FunctionTok{labs}\NormalTok{(}\AttributeTok{x =} \StringTok{"Index"}\NormalTok{,}
       \AttributeTok{title =} \StringTok{"Residual Time Sequence Plot for xplode Data"}\NormalTok{,}
       \AttributeTok{subtitle =} \StringTok{"by Jerry Yu"}\NormalTok{) }\SpecialCharTok{+}
  \FunctionTok{geom\_hline}\NormalTok{(}\AttributeTok{yintercept =} \DecValTok{0}\NormalTok{,}
             \AttributeTok{color =} \StringTok{"darkblue"}\NormalTok{,}
             \AttributeTok{linetype =} \StringTok{"dotdash"}\NormalTok{)}
\end{Highlighting}
\end{Shaded}

\includegraphics{YuHW03ST430_files/figure-latex/index2-1.pdf}

\hypertarget{f.-conduct-durbin-watson-test.}{%
\subsection{f.~Conduct Durbin-Watson
Test.}\label{f.-conduct-durbin-watson-test.}}

\begin{Shaded}
\begin{Highlighting}[]
\FunctionTok{dwtest}\NormalTok{(xplodem)}
\end{Highlighting}
\end{Shaded}

\begin{verbatim}
## 
##  Durbin-Watson test
## 
## data:  xplodem
## DW = 0.89573, p-value = 0.0008088
## alternative hypothesis: true autocorrelation is greater than 0
\end{verbatim}

\begin{Shaded}
\begin{Highlighting}[]
\NormalTok{xplodemw }\OtherTok{\textless{}{-}}\FunctionTok{durbinWatsonTest}\NormalTok{(xplodem)}
\NormalTok{xplodemw}
\end{Highlighting}
\end{Shaded}

\begin{verbatim}
##  lag Autocorrelation D-W Statistic p-value
##    1       0.3594126     0.8957278   0.008
##  Alternative hypothesis: rho != 0
\end{verbatim}

\hypertarget{g.-outlier-deduction-test.-plot-standardized-residuals-versus-fitted-values}{%
\subsection{g. outlier deduction test. {[}Plot standardized Residuals
versus fitted
values{]}}\label{g.-outlier-deduction-test.-plot-standardized-residuals-versus-fitted-values}}

\begin{Shaded}
\begin{Highlighting}[]
\NormalTok{xplodemrs }\OtherTok{\textless{}{-}} \FunctionTok{add\_column}\NormalTok{(xplodemr, }\StringTok{"rstandardized"}\OtherTok{=}\FunctionTok{rstandard}\NormalTok{(xplodem))}

\FunctionTok{ggplot}\NormalTok{(xplodemrs, }\FunctionTok{aes}\NormalTok{(}\AttributeTok{x =}\NormalTok{ fit, }\AttributeTok{y =}\NormalTok{ rstandardized)) }\SpecialCharTok{+}
  \FunctionTok{geom\_point}\NormalTok{(}\AttributeTok{color =} \StringTok{"lightgreen"}\NormalTok{) }\SpecialCharTok{+}
  \FunctionTok{labs}\NormalTok{(}\AttributeTok{x =} \StringTok{"Index"}\NormalTok{,}
       \AttributeTok{title =} \StringTok{"Outlier Detection Plot with Standarized Residuals vs Fit for Xplode Data"}\NormalTok{,}
       \AttributeTok{subtitle =} \StringTok{"by Jerry Yu"}\NormalTok{) }\SpecialCharTok{+}
  \FunctionTok{geom\_hline}\NormalTok{(}\AttributeTok{yintercept =} \DecValTok{0}\NormalTok{,}
             \AttributeTok{color =} \StringTok{"darkblue"}\NormalTok{,}
             \AttributeTok{linetype =} \StringTok{"solid"}\NormalTok{) }\SpecialCharTok{+}
    \FunctionTok{geom\_hline}\NormalTok{(}\AttributeTok{yintercept =} \SpecialCharTok{{-}}\DecValTok{2}\NormalTok{,}
             \AttributeTok{color =} \StringTok{"blue"}\NormalTok{,}
             \AttributeTok{linetype =} \StringTok{"dashed"}\NormalTok{)}\SpecialCharTok{+}
  \FunctionTok{geom\_hline}\NormalTok{(}\AttributeTok{yintercept =} \DecValTok{2}\NormalTok{,}
             \AttributeTok{color =} \StringTok{"blue"}\NormalTok{,}
             \AttributeTok{linetype =} \StringTok{"dashed"}\NormalTok{)}
\end{Highlighting}
\end{Shaded}

\includegraphics{YuHW03ST430_files/figure-latex/unnamed-chunk-3-1.pdf}

\begin{Shaded}
\begin{Highlighting}[]
\NormalTok{xplodemro }\OtherTok{\textless{}{-}} \FunctionTok{filter}\NormalTok{(xplodemrs,}\FunctionTok{abs}\NormalTok{(rstandardized) }\SpecialCharTok{\textgreater{}}\DecValTok{2}\NormalTok{)}
\NormalTok{xplodemro}
\end{Highlighting}
\end{Shaded}

\begin{verbatim}
## # A tibble: 1 x 3
##     fit resid rstandardized
##   <dbl> <dbl>         <dbl>
## 1  173.  19.4          2.80
\end{verbatim}

We have 1 outlier where the fit = \texttt{r\ xplodemro{[}{[}1,1{]}{]}}.

\hypertarget{i.-give-a-qq-plot-of-the-residuals.-normality-of-error-terms.}{%
\subsection{i. Give a QQ-plot of the residuals. Normality of error
terms.}\label{i.-give-a-qq-plot-of-the-residuals.-normality-of-error-terms.}}

\begin{Shaded}
\begin{Highlighting}[]
\FunctionTok{ggplot}\NormalTok{(}\AttributeTok{data =}\NormalTok{ xplodemrs, }\FunctionTok{aes}\NormalTok{(}\AttributeTok{sample =}\NormalTok{ resid))}\SpecialCharTok{+}
  \FunctionTok{geom\_qq}\NormalTok{( }\AttributeTok{color=}\StringTok{"coral"}\NormalTok{)}\SpecialCharTok{+}
  \FunctionTok{geom\_qq\_line}\NormalTok{( }\AttributeTok{color=}\StringTok{"turquoise"}\NormalTok{)}\SpecialCharTok{+}
  \FunctionTok{labs}\NormalTok{(}
    \AttributeTok{title =} \FunctionTok{paste}\NormalTok{(}\StringTok{"QQ Plot of Residuals for Xplode Linear Regression Model"}\NormalTok{),}
    \AttributeTok{subtitle =} \StringTok{"by Jerry Yu"}
\NormalTok{  ) }\SpecialCharTok{+}
  \FunctionTok{xlab}\NormalTok{(}\StringTok{"Theoretical Quantiles"}\NormalTok{)}\SpecialCharTok{+}
  \FunctionTok{ylab}\NormalTok{(}\StringTok{"Sample Quantiles"}\NormalTok{)}
\end{Highlighting}
\end{Shaded}

\includegraphics{YuHW03ST430_files/figure-latex/qq2-1.pdf}

\hypertarget{j.-conduct-a-shapiro-wilk-test-on-the-residuals.-give-the-p-value-for-this-test-and-explain-what-this-means-in-terms-of-our-model-assumptions.}{%
\subsection{j. Conduct a Shapiro-Wilk Test on the residuals. Give the
p-value for this test and explain what this means in terms of our model
assumptions.}\label{j.-conduct-a-shapiro-wilk-test-on-the-residuals.-give-the-p-value-for-this-test-and-explain-what-this-means-in-terms-of-our-model-assumptions.}}

\begin{Shaded}
\begin{Highlighting}[]
\NormalTok{xplodeshap }\OtherTok{\textless{}{-}} \FunctionTok{shapiro.test}\NormalTok{(xplodemrs}\SpecialCharTok{$}\NormalTok{resid)}
\NormalTok{xplodeshap}
\end{Highlighting}
\end{Shaded}

\begin{verbatim}
## 
##  Shapiro-Wilk normality test
## 
## data:  xplodemrs$resid
## W = 0.92354, p-value = 0.09004
\end{verbatim}

\begin{itemize}
\item
  H0: The random error is normally distributed
\item
  Ha: The random error is not normally distributed
\item
  p value: 0.0900395.
\end{itemize}

As p \textgreater{} 0.05, we fail to reject H0 at \(\alpha\) = 0.05 and
conclude that there is no statistically significant evidence that the
random error is not normally distributed.

\hypertarget{k.-give-the-anova-table-for-this-regression-analysis.-based-on-your-anova-table-is-the-linear-relationship-between-x-and-y-statistically-significant-be-sure-to-give-an-appropriate-test-statistic-its-associated-degrees-of-freedom-and-the-p-value.}{%
\subsection{k. Give the ANOVA Table for this regression analysis. Based
on your ANOVA table, is the linear relationship between X and Y
statistically significant? Be sure to give an appropriate test
statistic, its associated degrees of freedom, and the
p-value.}\label{k.-give-the-anova-table-for-this-regression-analysis.-based-on-your-anova-table-is-the-linear-relationship-between-x-and-y-statistically-significant-be-sure-to-give-an-appropriate-test-statistic-its-associated-degrees-of-freedom-and-the-p-value.}}

\begin{Shaded}
\begin{Highlighting}[]
\NormalTok{xplodema }\OtherTok{\textless{}{-}} \FunctionTok{anova}\NormalTok{(xplodem)}

\NormalTok{xplodema}
\end{Highlighting}
\end{Shaded}

\begin{verbatim}
## Analysis of Variance Table
## 
## Response: Galvonometer
##           Df  Sum Sq Mean Sq F value    Pr(>F)    
## Wire.Area  1 22749.5 22749.5   422.6 6.386e-15 ***
## Residuals 20  1076.6    53.8                      
## ---
## Signif. codes:  0 '***' 0.001 '**' 0.01 '*' 0.05 '.' 0.1 ' ' 1
\end{verbatim}

\begin{itemize}
\item
  H0: There is no linear relationship between Wire Area (1/100,000 in)
  and Deflection of Galvonometer in mm (\(\beta_1\) =0)
\item
  H0: There is a linear relationship between Wire Area (1/100,000 in)
  and Deflection of Galvonometer in mm (\(\beta_1\) ne 0)
\item
  Test Statistic (F value): 422.6036859
\item
  Degrees of Freedom: 1 for the model (Area of Wires (1/100,000 in)),
  and 20 for the error.
\item
  P value: \ensuremath{6.386449\times 10^{-15}}
\end{itemize}

As p \textless{} 0.05, we reject H0 at \(\alpha\) = 0.05 and conclude
that there is statistically significant evidence for a linear
relationship between Wire Area (1/100,000 in) and Deflection of
Galvonometer in mm.

\begin{Shaded}
\begin{Highlighting}[]
\NormalTok{surg.data }\OtherTok{\textless{}{-}} \FunctionTok{read.table}\NormalTok{(}
  \StringTok{"Datasets/Surgical Unit.txt"}\NormalTok{,}
  \AttributeTok{header =} \ConstantTok{FALSE}\NormalTok{,}
  \AttributeTok{col.names =} \FunctionTok{c}\NormalTok{(}
    \StringTok{"clot"}\NormalTok{,}
    \StringTok{"PI"}\NormalTok{,}
    \StringTok{"enzy"}\NormalTok{,}
    \StringTok{"liver"}\NormalTok{,}
    \StringTok{"age"}\NormalTok{,}
    \StringTok{"gender"}\NormalTok{,}
    \StringTok{"mod\_use"}\NormalTok{,}
    \StringTok{"heavy\_use"}\NormalTok{,}
    \StringTok{"sur\_time"}\NormalTok{,}
    \StringTok{"ln\_sur\_time"}
\NormalTok{  )}
\NormalTok{)}

\FunctionTok{attach}\NormalTok{(surg.data)}
\NormalTok{gender }\OtherTok{=} \FunctionTok{factor}\NormalTok{(gender)}

\NormalTok{surg.datam }\OtherTok{\textless{}{-}} \FunctionTok{lm}\NormalTok{(sur\_time}\SpecialCharTok{\textasciitilde{}}\NormalTok{clot,}\AttributeTok{data=}\NormalTok{surg.data)}

\NormalTok{surg.datam}
\end{Highlighting}
\end{Shaded}

\begin{verbatim}
## 
## Call:
## lm(formula = sur_time ~ clot, data = surg.data)
## 
## Coefficients:
## (Intercept)         clot  
##      205.26        85.91
\end{verbatim}

\begin{Shaded}
\begin{Highlighting}[]
\NormalTok{surg.dataa }\OtherTok{\textless{}{-}} \FunctionTok{anova}\NormalTok{(surg.datam)}

\NormalTok{surg.dataa}
\end{Highlighting}
\end{Shaded}

\begin{verbatim}
## Analysis of Variance Table
## 
## Response: sur_time
##           Df  Sum Sq Mean Sq F value  Pr(>F)  
## clot       1 1005152 1005152  7.0974 0.01025 *
## Residuals 52 7364369  141622                  
## ---
## Signif. codes:  0 '***' 0.001 '**' 0.01 '*' 0.05 '.' 0.1 ' ' 1
\end{verbatim}

\hypertarget{question-3}{%
\section{Question 3}\label{question-3}}

\hypertarget{a.-based-on-your-anova-table-is-the-linear-relationship-between-x-and-y-statistically-significant-be-sure-to-give-an-appropriate-null-and-alternate-hypothesis-test-statistic-its-associated-degrees-of-freedom-and-the-p-value.}{%
\subsection{a. Based on your ANOVA table, is the linear relationship
between X and Y statistically significant? Be sure to give an
appropriate null and alternate hypothesis, test statistic, its
associated degrees of freedom, and the
p-value.}\label{a.-based-on-your-anova-table-is-the-linear-relationship-between-x-and-y-statistically-significant-be-sure-to-give-an-appropriate-null-and-alternate-hypothesis-test-statistic-its-associated-degrees-of-freedom-and-the-p-value.}}

\begin{itemize}
\item
  H0: There is no linear relationship between survival time and blood
  clotting score (\(\beta_1\) =0)
\item
  HA: There is a linear relationship between survival time and blood
  clotting score (\(\beta_1\) ne 0)
\item
  Test Statistic (F value): 7.097402
\item
  Degrees of Freedom: 1 for the model (Blood-clotting score), and 52 for
  the error.
\item
  P value: 0.0102549
\end{itemize}

As p \textless{} 0.05, we reject H0 at \(\alpha\) = 0.05 and conclude
that there is statistically significant evidence for a linear
relationship between blood clotting score and survival time.

\hypertarget{b.-give-a-scatter-plot-of-clot-vs.-sur_time-with-the}{%
\subsection{b. Give a scatter plot of clot vs.~sur\_time, with
the}\label{b.-give-a-scatter-plot-of-clot-vs.-sur_time-with-the}}

regression line. Comment about linearity

\begin{Shaded}
\begin{Highlighting}[]
\FunctionTok{ggplot}\NormalTok{(surg.data,}\FunctionTok{aes}\NormalTok{(}\AttributeTok{x=}\NormalTok{clot,}\AttributeTok{y=}\NormalTok{sur\_time))}\SpecialCharTok{+}
  \FunctionTok{geom\_jitter}\NormalTok{(}\AttributeTok{color=}\StringTok{"turquoise"}\NormalTok{)}\SpecialCharTok{+}
  \FunctionTok{geom\_smooth}\NormalTok{(}\AttributeTok{method=}\StringTok{\textquotesingle{}lm\textquotesingle{}}\NormalTok{, }\AttributeTok{formula=}\NormalTok{ y}\SpecialCharTok{\textasciitilde{}}\NormalTok{x,}
              \AttributeTok{se=}\ConstantTok{FALSE}\NormalTok{,}
              \AttributeTok{show.legend=}\ConstantTok{TRUE}\NormalTok{)}\SpecialCharTok{+}
    \FunctionTok{stat\_poly\_eq}\NormalTok{(}\AttributeTok{eq.with.lhs =} \StringTok{"italic(hat(y))\textasciitilde{}\textasciigrave{}=\textasciigrave{}\textasciitilde{}"}\NormalTok{,}
               \FunctionTok{use\_label}\NormalTok{(}\FunctionTok{c}\NormalTok{(}\StringTok{"eq"}\NormalTok{, }\StringTok{"R2"}\NormalTok{)))}\SpecialCharTok{+}
  \FunctionTok{labs}\NormalTok{(}\AttributeTok{title =} \FunctionTok{paste}\NormalTok{(}\StringTok{"Scatterplot of Clotting Score and Survival Time }\SpecialCharTok{\textbackslash{}n}\StringTok{ with Linear Regression Line and Equation"}\NormalTok{),}
         \AttributeTok{subtitle =} \StringTok{"by Jerry Yu"}\NormalTok{)}
\end{Highlighting}
\end{Shaded}

\includegraphics{YuHW03ST430_files/figure-latex/scatter3-1.pdf}

I would say that the distributions of residuals above and below the
regression line look somewhat even. However there is a noticeable funnel
pattern, but that is indicative of heteroscedasticity, not
non-linearity.

\hypertarget{c.-give-the-residual-plot-residuals-vs.-fitted-values.-test-for-non-linear-and-non-constant-variance.-1}{%
\subsection{c.~Give the Residual Plot (residuals vs.~fitted values).
Test for Non-Linear and Non-constant
variance.}\label{c.-give-the-residual-plot-residuals-vs.-fitted-values.-test-for-non-linear-and-non-constant-variance.-1}}

\begin{Shaded}
\begin{Highlighting}[]
\NormalTok{surg.datamr }\OtherTok{\textless{}{-}} \FunctionTok{tibble}\NormalTok{(}
  \StringTok{"fit"} \OtherTok{=}\NormalTok{ surg.datam}\SpecialCharTok{$}\NormalTok{fitted.values,}
  \StringTok{"resid"} \OtherTok{=}\NormalTok{ surg.datam}\SpecialCharTok{$}\NormalTok{residuals}
\NormalTok{)}

\FunctionTok{ggplot}\NormalTok{(surg.datamr,}\FunctionTok{aes}\NormalTok{(}\AttributeTok{x=}\NormalTok{fit,}\AttributeTok{y=}\NormalTok{resid))}\SpecialCharTok{+}
  \FunctionTok{geom\_jitter}\NormalTok{(}\AttributeTok{color=}\StringTok{"salmon"}\NormalTok{)}\SpecialCharTok{+}
  \FunctionTok{geom\_hline}\NormalTok{(}\AttributeTok{yintercept =} \DecValTok{0}\NormalTok{, }\AttributeTok{linetype=}\StringTok{"dotted"}\NormalTok{)}\SpecialCharTok{+}
  \FunctionTok{labs}\NormalTok{(}\AttributeTok{title =} \FunctionTok{paste}\NormalTok{(}\StringTok{"Residuals Versus Fitted Values for the surg.data Data Set"}\NormalTok{),}
         \AttributeTok{subtitle =} \StringTok{"by Jerry Yu"}\NormalTok{)}\SpecialCharTok{+}
  \FunctionTok{theme}\NormalTok{(}\AttributeTok{plot.title =} \FunctionTok{element\_text}\NormalTok{(}\AttributeTok{size =} \DecValTok{14}\NormalTok{))}
\end{Highlighting}
\end{Shaded}

\includegraphics{YuHW03ST430_files/figure-latex/resid3-1.pdf}

\hypertarget{d.-conduct-breusch-pagan-test-for-the-constancy-of-the-error-variance.-be-sure-to-give-an-appropriate-null-and-alternate-hypothesis-test-statistic-its-associated-degrees-of-freedom-and-the-p-value.}{%
\subsection{d.~Conduct Breusch-Pagan Test for the constancy of the error
variance. Be sure to give an appropriate null and alternate hypothesis,
test statistic, its associated degrees of freedom, and the
p-value.}\label{d.-conduct-breusch-pagan-test-for-the-constancy-of-the-error-variance.-be-sure-to-give-an-appropriate-null-and-alternate-hypothesis-test-statistic-its-associated-degrees-of-freedom-and-the-p-value.}}

\begin{Shaded}
\begin{Highlighting}[]
\NormalTok{surg.datambp }\OtherTok{\textless{}{-}} \FunctionTok{bptest}\NormalTok{(surg.datam,}\AttributeTok{studentize =} \ConstantTok{FALSE}\NormalTok{)}
\FunctionTok{ncvTest}\NormalTok{(surg.datam)}
\end{Highlighting}
\end{Shaded}

\begin{verbatim}
## Non-constant Variance Score Test 
## Variance formula: ~ fitted.values 
## Chisquare = 14.44279, Df = 1, p = 0.00014448
\end{verbatim}

\begin{itemize}
\tightlist
\item
  H0: Equal Variance Among Errors
\item
  HA: Unequal Variance Among Errors
\item
  Degree of Freedom: 1
\item
  P Value: \ensuremath{1.4448211\times 10^{-4}}
\end{itemize}

\hypertarget{e.-index-plot-to-test-for-independence-of-errors.-1}{%
\subsection{e. Index Plot to test for Independence of
errors.}\label{e.-index-plot-to-test-for-independence-of-errors.-1}}

\begin{Shaded}
\begin{Highlighting}[]
\FunctionTok{ggplot}\NormalTok{(surg.datamr, }\FunctionTok{aes}\NormalTok{(}\AttributeTok{x =} \DecValTok{1}\SpecialCharTok{:}\FunctionTok{length}\NormalTok{(resid), }\AttributeTok{y =}\NormalTok{ resid)) }\SpecialCharTok{+}
  \FunctionTok{geom\_point}\NormalTok{(}\AttributeTok{color =} \StringTok{"aquamarine"}\NormalTok{) }\SpecialCharTok{+}
  \FunctionTok{labs}\NormalTok{(}\AttributeTok{x =} \StringTok{"Index"}\NormalTok{,}
       \AttributeTok{title =} \StringTok{"Residual Time Sequence Plot for Surgical Data"}\NormalTok{,}
       \AttributeTok{subtitle =} \StringTok{"by Jerry Yu"}\NormalTok{) }\SpecialCharTok{+}
  \FunctionTok{geom\_hline}\NormalTok{(}\AttributeTok{yintercept =} \DecValTok{0}\NormalTok{,}
             \AttributeTok{color =} \StringTok{"darkblue"}\NormalTok{,}
             \AttributeTok{linetype =} \StringTok{"dotdash"}\NormalTok{)}
\end{Highlighting}
\end{Shaded}

\includegraphics{YuHW03ST430_files/figure-latex/index3-1.pdf}

\hypertarget{f.-conduct-durbin-watson-test.-be-sure-to-give-an-appropriate-null-and-alternate-hypothesis-test-statistic-and-the-p-value.}{%
\subsection{f.~Conduct Durbin-Watson Test. Be sure to give an
appropriate null and alternate hypothesis, test statistic and the
p-value.}\label{f.-conduct-durbin-watson-test.-be-sure-to-give-an-appropriate-null-and-alternate-hypothesis-test-statistic-and-the-p-value.}}

\begin{Shaded}
\begin{Highlighting}[]
\FunctionTok{dwtest}\NormalTok{(surg.datam)}
\end{Highlighting}
\end{Shaded}

\begin{verbatim}
## 
##  Durbin-Watson test
## 
## data:  surg.datam
## DW = 2.3034, p-value = 0.8724
## alternative hypothesis: true autocorrelation is greater than 0
\end{verbatim}

\begin{Shaded}
\begin{Highlighting}[]
\NormalTok{surg.datamw }\OtherTok{\textless{}{-}}\FunctionTok{durbinWatsonTest}\NormalTok{(surg.datam)}
\NormalTok{surg.datamw}
\end{Highlighting}
\end{Shaded}

\begin{verbatim}
##  lag Autocorrelation D-W Statistic p-value
##    1      -0.1587264      2.303382   0.228
##  Alternative hypothesis: rho != 0
\end{verbatim}

\begin{itemize}
\item
  H0: Errors are uncorrelated over time
\item
  HA: Errors are correlated (either positive or negative). I used the
  \texttt{car} test where the alternative hypothesis is 2 sided
\item
  Test Statistic: 2.3033819
\item
  p value: 0.228
\end{itemize}

\hypertarget{g.-outlier-deduction-test-plot-standardized-residuals-versus-fitted-values}{%
\subsection{g. Outlier deduction test {[}Plot standardized Residuals
versus fitted
values{]}}\label{g.-outlier-deduction-test-plot-standardized-residuals-versus-fitted-values}}

\begin{Shaded}
\begin{Highlighting}[]
\NormalTok{surg.datamrs }\OtherTok{\textless{}{-}} \FunctionTok{add\_column}\NormalTok{(surg.datamr, }\StringTok{"rstandardized"}\OtherTok{=}\FunctionTok{rstandard}\NormalTok{(surg.datam))}

\FunctionTok{ggplot}\NormalTok{(surg.datamrs, }\FunctionTok{aes}\NormalTok{(}\AttributeTok{x =}\NormalTok{ fit, }\AttributeTok{y =}\NormalTok{ rstandardized)) }\SpecialCharTok{+}
  \FunctionTok{geom\_point}\NormalTok{(}\AttributeTok{color =} \StringTok{"lightgreen"}\NormalTok{) }\SpecialCharTok{+}
  \FunctionTok{labs}\NormalTok{(}\AttributeTok{x =} \StringTok{"Index"}\NormalTok{,}
       \AttributeTok{title =} \StringTok{"Outlier Detection Plot with Standarized Residuals vs Fit"}\NormalTok{,}
       \AttributeTok{subtitle =} \StringTok{"by Jerry Yu"}\NormalTok{) }\SpecialCharTok{+}
  \FunctionTok{geom\_hline}\NormalTok{(}\AttributeTok{yintercept =} \DecValTok{0}\NormalTok{,}
             \AttributeTok{color =} \StringTok{"darkblue"}\NormalTok{,}
             \AttributeTok{linetype =} \StringTok{"solid"}\NormalTok{) }\SpecialCharTok{+}
    \FunctionTok{geom\_hline}\NormalTok{(}\AttributeTok{yintercept =} \SpecialCharTok{{-}}\DecValTok{2}\NormalTok{,}
             \AttributeTok{color =} \StringTok{"blue"}\NormalTok{,}
             \AttributeTok{linetype =} \StringTok{"dashed"}\NormalTok{)}\SpecialCharTok{+}
  \FunctionTok{geom\_hline}\NormalTok{(}\AttributeTok{yintercept =} \DecValTok{2}\NormalTok{,}
             \AttributeTok{color =} \StringTok{"blue"}\NormalTok{,}
             \AttributeTok{linetype =} \StringTok{"dashed"}\NormalTok{)}
\end{Highlighting}
\end{Shaded}

\includegraphics{YuHW03ST430_files/figure-latex/out3-1.pdf}

\begin{Shaded}
\begin{Highlighting}[]
\NormalTok{surg.datamro }\OtherTok{\textless{}{-}} \FunctionTok{filter}\NormalTok{(surg.datamrs,}\FunctionTok{abs}\NormalTok{(rstandardized) }\SpecialCharTok{\textgreater{}}\DecValTok{2}\NormalTok{)}
\NormalTok{surg.datamro}
\end{Highlighting}
\end{Shaded}

\begin{verbatim}
## # A tibble: 4 x 3
##     fit resid rstandardized
##   <dbl> <dbl>         <dbl>
## 1  875. 1468.          4.00
## 2  704.  869.          2.33
## 3 1167.  798.          2.42
## 4  953. -772.         -2.14
\end{verbatim}

We have 4 outliers, shown in the table \texttt{surg.datamro}.

\hypertarget{h.-give-a-histogram-of-the-residuals-and-the-density-curve.-comment-about-the-distribution-of-residuals.}{%
\subsection{h. Give a Histogram of the residuals and the density curve.
Comment about the distribution of
residuals.}\label{h.-give-a-histogram-of-the-residuals-and-the-density-curve.-comment-about-the-distribution-of-residuals.}}

\begin{Shaded}
\begin{Highlighting}[]
\FunctionTok{ggplot}\NormalTok{(}\AttributeTok{data =}\NormalTok{ surg.datamrs, }\FunctionTok{aes}\NormalTok{(}\AttributeTok{x =}\NormalTok{ resid, }\AttributeTok{y =} \FunctionTok{after\_stat}\NormalTok{(density))) }\SpecialCharTok{+}
  \FunctionTok{geom\_histogram}\NormalTok{(}\AttributeTok{fill =} \StringTok{"lightblue"}\NormalTok{) }\SpecialCharTok{+}
  \FunctionTok{geom\_density}\NormalTok{(}\AttributeTok{color =} \StringTok{"darkgreen"}\NormalTok{) }\SpecialCharTok{+}
  \FunctionTok{labs}\NormalTok{(}
    \AttributeTok{title =} \FunctionTok{paste}\NormalTok{(}\StringTok{"Histogram and Density Plot of Residuals for Data Set surg.data"}\NormalTok{),}
    \AttributeTok{subtitle =} \StringTok{"by Jerry Yu"}
\NormalTok{  ) }\SpecialCharTok{+}
  \FunctionTok{ylab}\NormalTok{(}\StringTok{"Density of Residuals"}\NormalTok{)}
\end{Highlighting}
\end{Shaded}

\begin{verbatim}
## `stat_bin()` using `bins = 30`. Pick better value with `binwidth`.
\end{verbatim}

\includegraphics{YuHW03ST430_files/figure-latex/density3-1.pdf}

There seems to be a slight right skew in the data, 3 of the 4 outliers
detected in part g are clearly visible.

\hypertarget{i.-give-a-qq-plot-of-the-residuals-to-test-for-normality-of-error-terms.-comment-about-the-distribution-of-residuals.}{%
\subsection{i. Give a QQ-plot of the residuals to test for normality of
error terms. Comment about the distribution of
residuals.}\label{i.-give-a-qq-plot-of-the-residuals-to-test-for-normality-of-error-terms.-comment-about-the-distribution-of-residuals.}}

\begin{Shaded}
\begin{Highlighting}[]
\FunctionTok{ggplot}\NormalTok{(}\AttributeTok{data =}\NormalTok{ surg.datamrs, }\FunctionTok{aes}\NormalTok{(}\AttributeTok{sample =}\NormalTok{ resid))}\SpecialCharTok{+}
  \FunctionTok{geom\_qq}\NormalTok{( }\AttributeTok{color=}\StringTok{"coral"}\NormalTok{)}\SpecialCharTok{+}
  \FunctionTok{geom\_qq\_line}\NormalTok{( }\AttributeTok{color=}\StringTok{"turquoise"}\NormalTok{)}\SpecialCharTok{+}
  \FunctionTok{labs}\NormalTok{(}
    \AttributeTok{title =} \FunctionTok{paste}\NormalTok{(}\StringTok{"QQ Plot of Residuals for surg.data Linear Regression Model"}\NormalTok{),}
    \AttributeTok{subtitle =} \StringTok{"by Jerry Yu"}
\NormalTok{  ) }\SpecialCharTok{+}
  \FunctionTok{xlab}\NormalTok{(}\StringTok{"Theoretical Quantiles"}\NormalTok{)}\SpecialCharTok{+}
  \FunctionTok{ylab}\NormalTok{(}\StringTok{"Sample Quantiles"}\NormalTok{)}
\end{Highlighting}
\end{Shaded}

\includegraphics{YuHW03ST430_files/figure-latex/qq3-1.pdf}

The data visually does not look normal, and demonstrates strong signs of
heteroscedasticity.The residuals at the end seem to all deviate from the
line.

\hypertarget{j.-conduct-a-shapiro-wilk-test-on-the-residuals.-be-sure-to-give-an-appropriate-null-and-alternate-hypothesis-test-statistic-and-the-p-value.-give-the-p-value-for-this-test-and-explain-what-this-means-in-terms-of-our-model-assumptions}{%
\subsection{j. Conduct a Shapiro-Wilk Test on the residuals. Be sure to
give an appropriate null and alternate hypothesis, test statistic and
the p-value. Give the p-value for this test and explain what this means
in terms of our model
assumptions}\label{j.-conduct-a-shapiro-wilk-test-on-the-residuals.-be-sure-to-give-an-appropriate-null-and-alternate-hypothesis-test-statistic-and-the-p-value.-give-the-p-value-for-this-test-and-explain-what-this-means-in-terms-of-our-model-assumptions}}

\begin{Shaded}
\begin{Highlighting}[]
\NormalTok{surg.datasp }\OtherTok{\textless{}{-}} \FunctionTok{shapiro.test}\NormalTok{(surg.datamrs}\SpecialCharTok{$}\NormalTok{resid)}
\NormalTok{surg.datasp}
\end{Highlighting}
\end{Shaded}

\begin{verbatim}
## 
##  Shapiro-Wilk normality test
## 
## data:  surg.datamrs$resid
## W = 0.87793, p-value = 5.348e-05
\end{verbatim}

\begin{itemize}
\tightlist
\item
  H0: The random error is normally distributed
\item
  Ha: The random error is not normally distributed
\item
  Test Statistic: 0.8779272
\item
  p value: \ensuremath{5.3476597\times 10^{-5}}
\end{itemize}

As p \textless{} 0.05, we reject H0 at \(\alpha\) = 0.05 and conclude
that there is statistically significant evidence that the random error
is not normally distributed. This means that we cannot assume that the
random error is normally distributed for our model, and thus a linear
regression without transformation of the data is not advised.

H0: \(\frac{\beta_{software2*Sales Last Quarter}}{2}\) +
\(\frac{\beta_{software2}{2}\) =
\(\beta_{software2*Sales Last Quarter}\) + \(\beta_{software2}\)

\end{document}
